\documentclass[]{book}
\usepackage{lmodern}
\usepackage{amssymb,amsmath}
\usepackage{ifxetex,ifluatex}
\usepackage{fixltx2e} % provides \textsubscript
\ifnum 0\ifxetex 1\fi\ifluatex 1\fi=0 % if pdftex
  \usepackage[T1]{fontenc}
  \usepackage[utf8]{inputenc}
\else % if luatex or xelatex
  \ifxetex
    \usepackage{mathspec}
  \else
    \usepackage{fontspec}
  \fi
  \defaultfontfeatures{Ligatures=TeX,Scale=MatchLowercase}
\fi
% use upquote if available, for straight quotes in verbatim environments
\IfFileExists{upquote.sty}{\usepackage{upquote}}{}
% use microtype if available
\IfFileExists{microtype.sty}{%
\usepackage[]{microtype}
\UseMicrotypeSet[protrusion]{basicmath} % disable protrusion for tt fonts
}{}
\PassOptionsToPackage{hyphens}{url} % url is loaded by hyperref
\usepackage[unicode=true]{hyperref}
\hypersetup{
            pdftitle={R},
            pdfauthor={Bill Last Updated:},
            pdfborder={0 0 0},
            breaklinks=true}
\urlstyle{same}  % don't use monospace font for urls
\usepackage{natbib}
\bibliographystyle{apalike}
\usepackage{color}
\usepackage{fancyvrb}
\newcommand{\VerbBar}{|}
\newcommand{\VERB}{\Verb[commandchars=\\\{\}]}
\DefineVerbatimEnvironment{Highlighting}{Verbatim}{commandchars=\\\{\}}
% Add ',fontsize=\small' for more characters per line
\usepackage{framed}
\definecolor{shadecolor}{RGB}{248,248,248}
\newenvironment{Shaded}{\begin{snugshade}}{\end{snugshade}}
\newcommand{\KeywordTok}[1]{\textcolor[rgb]{0.13,0.29,0.53}{\textbf{#1}}}
\newcommand{\DataTypeTok}[1]{\textcolor[rgb]{0.13,0.29,0.53}{#1}}
\newcommand{\DecValTok}[1]{\textcolor[rgb]{0.00,0.00,0.81}{#1}}
\newcommand{\BaseNTok}[1]{\textcolor[rgb]{0.00,0.00,0.81}{#1}}
\newcommand{\FloatTok}[1]{\textcolor[rgb]{0.00,0.00,0.81}{#1}}
\newcommand{\ConstantTok}[1]{\textcolor[rgb]{0.00,0.00,0.00}{#1}}
\newcommand{\CharTok}[1]{\textcolor[rgb]{0.31,0.60,0.02}{#1}}
\newcommand{\SpecialCharTok}[1]{\textcolor[rgb]{0.00,0.00,0.00}{#1}}
\newcommand{\StringTok}[1]{\textcolor[rgb]{0.31,0.60,0.02}{#1}}
\newcommand{\VerbatimStringTok}[1]{\textcolor[rgb]{0.31,0.60,0.02}{#1}}
\newcommand{\SpecialStringTok}[1]{\textcolor[rgb]{0.31,0.60,0.02}{#1}}
\newcommand{\ImportTok}[1]{#1}
\newcommand{\CommentTok}[1]{\textcolor[rgb]{0.56,0.35,0.01}{\textit{#1}}}
\newcommand{\DocumentationTok}[1]{\textcolor[rgb]{0.56,0.35,0.01}{\textbf{\textit{#1}}}}
\newcommand{\AnnotationTok}[1]{\textcolor[rgb]{0.56,0.35,0.01}{\textbf{\textit{#1}}}}
\newcommand{\CommentVarTok}[1]{\textcolor[rgb]{0.56,0.35,0.01}{\textbf{\textit{#1}}}}
\newcommand{\OtherTok}[1]{\textcolor[rgb]{0.56,0.35,0.01}{#1}}
\newcommand{\FunctionTok}[1]{\textcolor[rgb]{0.00,0.00,0.00}{#1}}
\newcommand{\VariableTok}[1]{\textcolor[rgb]{0.00,0.00,0.00}{#1}}
\newcommand{\ControlFlowTok}[1]{\textcolor[rgb]{0.13,0.29,0.53}{\textbf{#1}}}
\newcommand{\OperatorTok}[1]{\textcolor[rgb]{0.81,0.36,0.00}{\textbf{#1}}}
\newcommand{\BuiltInTok}[1]{#1}
\newcommand{\ExtensionTok}[1]{#1}
\newcommand{\PreprocessorTok}[1]{\textcolor[rgb]{0.56,0.35,0.01}{\textit{#1}}}
\newcommand{\AttributeTok}[1]{\textcolor[rgb]{0.77,0.63,0.00}{#1}}
\newcommand{\RegionMarkerTok}[1]{#1}
\newcommand{\InformationTok}[1]{\textcolor[rgb]{0.56,0.35,0.01}{\textbf{\textit{#1}}}}
\newcommand{\WarningTok}[1]{\textcolor[rgb]{0.56,0.35,0.01}{\textbf{\textit{#1}}}}
\newcommand{\AlertTok}[1]{\textcolor[rgb]{0.94,0.16,0.16}{#1}}
\newcommand{\ErrorTok}[1]{\textcolor[rgb]{0.64,0.00,0.00}{\textbf{#1}}}
\newcommand{\NormalTok}[1]{#1}
\usepackage{longtable,booktabs}
% Fix footnotes in tables (requires footnote package)
\IfFileExists{footnote.sty}{\usepackage{footnote}\makesavenoteenv{long table}}{}
\usepackage{graphicx,grffile}
\makeatletter
\def\maxwidth{\ifdim\Gin@nat@width>\linewidth\linewidth\else\Gin@nat@width\fi}
\def\maxheight{\ifdim\Gin@nat@height>\textheight\textheight\else\Gin@nat@height\fi}
\makeatother
% Scale images if necessary, so that they will not overflow the page
% margins by default, and it is still possible to overwrite the defaults
% using explicit options in \includegraphics[width, height, ...]{}
\setkeys{Gin}{width=\maxwidth,height=\maxheight,keepaspectratio}
\IfFileExists{parskip.sty}{%
\usepackage{parskip}
}{% else
\setlength{\parindent}{0pt}
\setlength{\parskip}{6pt plus 2pt minus 1pt}
}
\setlength{\emergencystretch}{3em}  % prevent overfull lines
\providecommand{\tightlist}{%
  \setlength{\itemsep}{0pt}\setlength{\parskip}{0pt}}
\setcounter{secnumdepth}{5}
% Redefines (sub)paragraphs to behave more like sections
\ifx\paragraph\undefined\else
\let\oldparagraph\paragraph
\renewcommand{\paragraph}[1]{\oldparagraph{#1}\mbox{}}
\fi
\ifx\subparagraph\undefined\else
\let\oldsubparagraph\subparagraph
\renewcommand{\subparagraph}[1]{\oldsubparagraph{#1}\mbox{}}
\fi

% set default figure placement to htbp
\makeatletter
\def\fps@figure{htbp}
\makeatother

\usepackage{booktabs}
\usepackage{amsthm}
\makeatletter
\def\thm@space@setup{%
  \thm@preskip=8pt plus 2pt minus 4pt
  \thm@postskip=\thm@preskip
}
\makeatother

\title{R}
\author{Bill Last Updated:}
\date{19 January, 2020}

\begin{document}
\maketitle

{
\setcounter{tocdepth}{1}
\tableofcontents
}
\chapter*{Preface: Motivation}\label{my-section}
\addcontentsline{toc}{chapter}{Preface: Motivation}

All the notes I have done here are about R. While I have tried my best,
probably there are still some typos and errors. Please feel free to let
me know in case you find one. Thank you!

\chapter{apply, lapply, sapply}\label{apply-lapply-sapply}

\section{apply}\label{apply}

\begin{Shaded}
\begin{Highlighting}[]
\NormalTok{m_trying <-}\StringTok{ }\KeywordTok{matrix}\NormalTok{(C<-(}\DecValTok{1}\OperatorTok{:}\DecValTok{10}\NormalTok{),}\DataTypeTok{nrow=}\DecValTok{2}\NormalTok{, }\DataTypeTok{ncol=}\DecValTok{5}\NormalTok{)}
\NormalTok{m_trying}
\end{Highlighting}
\end{Shaded}

\begin{verbatim}
##      [,1] [,2] [,3] [,4] [,5]
## [1,]    1    3    5    7    9
## [2,]    2    4    6    8   10
\end{verbatim}

\begin{Shaded}
\begin{Highlighting}[]
\NormalTok{## Operating on the columns}
\KeywordTok{apply}\NormalTok{(m_trying, }\DecValTok{2}\NormalTok{, sum)}
\end{Highlighting}
\end{Shaded}

\begin{verbatim}
## [1]  3  7 11 15 19
\end{verbatim}

\begin{Shaded}
\begin{Highlighting}[]
\NormalTok{## Operating on the rows}
\KeywordTok{apply}\NormalTok{(m_trying, }\DecValTok{1}\NormalTok{, sum)}
\end{Highlighting}
\end{Shaded}

\begin{verbatim}
## [1] 25 30
\end{verbatim}

\section{lapply}\label{lapply}

``lapply returns a list of the same length as X, each element of which
is the result of applying FUN to the corresponding element of X.''

lapply operates on lists. Thus, as we can see below, even if m\_trying
is not a list, each cell becomes a list.

\begin{Shaded}
\begin{Highlighting}[]
\NormalTok{results1<-}\KeywordTok{lapply}\NormalTok{(m_trying,sum)}
\KeywordTok{str}\NormalTok{(results1)}
\end{Highlighting}
\end{Shaded}

\begin{verbatim}
## List of 10
##  $ : int 1
##  $ : int 2
##  $ : int 3
##  $ : int 4
##  $ : int 5
##  $ : int 6
##  $ : int 7
##  $ : int 8
##  $ : int 9
##  $ : int 10
\end{verbatim}

\begin{Shaded}
\begin{Highlighting}[]
\KeywordTok{is.list}\NormalTok{(results1)}
\end{Highlighting}
\end{Shaded}

\begin{verbatim}
## [1] TRUE
\end{verbatim}

\section{sapply}\label{sapply}

``sapply() function takes list, vector or data frame as input and gives
output in vector or matrix.''

\begin{Shaded}
\begin{Highlighting}[]
\NormalTok{results2<-}\KeywordTok{sapply}\NormalTok{(m_trying, sum)}
\KeywordTok{str}\NormalTok{(results2)}
\end{Highlighting}
\end{Shaded}

\begin{verbatim}
##  int [1:10] 1 2 3 4 5 6 7 8 9 10
\end{verbatim}

\begin{Shaded}
\begin{Highlighting}[]
\KeywordTok{is.list}\NormalTok{(results2)}
\end{Highlighting}
\end{Shaded}

\begin{verbatim}
## [1] FALSE
\end{verbatim}

\begin{Shaded}
\begin{Highlighting}[]
\KeywordTok{is.matrix}\NormalTok{(results2)}
\end{Highlighting}
\end{Shaded}

\begin{verbatim}
## [1] FALSE
\end{verbatim}

\begin{Shaded}
\begin{Highlighting}[]
\KeywordTok{is.data.frame}\NormalTok{(results2)}
\end{Highlighting}
\end{Shaded}

\begin{verbatim}
## [1] FALSE
\end{verbatim}

\begin{Shaded}
\begin{Highlighting}[]
\KeywordTok{is.vector}\NormalTok{(results2)}
\end{Highlighting}
\end{Shaded}

\begin{verbatim}
## [1] TRUE
\end{verbatim}

\chapter{C}\label{c}

\begin{Shaded}
\begin{Highlighting}[]
\NormalTok{mydata1<-}\KeywordTok{matrix}\NormalTok{(}\KeywordTok{runif}\NormalTok{(}\DecValTok{4}\OperatorTok{*}\DecValTok{2}\NormalTok{),}\DecValTok{4}\NormalTok{,}\DecValTok{2}\NormalTok{)}
\NormalTok{mydata1}
\end{Highlighting}
\end{Shaded}

\begin{verbatim}
##           [,1]      [,2]
## [1,] 0.6718513 0.2906222
## [2,] 0.6817739 0.2613887
## [3,] 0.8843620 0.2274622
## [4,] 0.1204287 0.5684971
\end{verbatim}

\begin{Shaded}
\begin{Highlighting}[]
\KeywordTok{str}\NormalTok{(mydata1)}
\end{Highlighting}
\end{Shaded}

\begin{verbatim}
##  num [1:4, 1:2] 0.672 0.682 0.884 0.12 0.291 ...
\end{verbatim}

\begin{Shaded}
\begin{Highlighting}[]
\NormalTok{mydata2<-}\KeywordTok{c}\NormalTok{(mydata1)}
\NormalTok{mydata2}
\end{Highlighting}
\end{Shaded}

\begin{verbatim}
## [1] 0.6718513 0.6817739 0.8843620 0.1204287 0.2906222 0.2613887 0.2274622
## [8] 0.5684971
\end{verbatim}

\begin{Shaded}
\begin{Highlighting}[]
\KeywordTok{str}\NormalTok{(mydata2)}
\end{Highlighting}
\end{Shaded}

\begin{verbatim}
##  num [1:8] 0.672 0.682 0.884 0.12 0.291 ...
\end{verbatim}

\bibliography{book.bib,packages.bib}

\end{document}
